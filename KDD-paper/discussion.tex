\section{Discussion}

In settings where there are many more items than the typical user
can consume, unobserved consumption is likely explained by finite
attention, as opposed to an active dislike for the associated
content. Likewise, a user's choice in selecting a particular set of
items amongst the many available options is a relatively strong
indicator of her interests. BPF captures these features of sparse user
data via the Poisson likelihood, which appropriately balances strong
signals of consumption with weaker signals of unobserved activity.

Conveniently, the same Poisson likelihood also leads to
computationally efficient inference on sparse data sets, as it requires
evaluation of only the consumed user-item pairs, which comprise a
small fraction of all possible observations. This avoids the issue
faced by traditional matrix factorization in down-weighting or sampling
negative examples during training. In addition to this computational
advantage, BPF empirically outperforms classical MF across a wide array
of data sets---from movies to music to scientific articles---in
recommending relevant content to users.
%Furthermore, we have shown that BPF is a variant of LDA conditioned on
%each user's activity, a previously unknown connection.

%Future work includes exploring modifications to BPF to incorporate
%additional features---e.g., user and item metadata---as well as
%stochastic inference to scale to massive data sets.
